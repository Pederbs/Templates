%& -job-name=Simple_Article
\documentclass[12pt]{article}
\usepackage[a4paper, margin=2.5cm]{geometry}
\usepackage{graphicx} % Required for inserting images
\usepackage{url}
\usepackage{hyperref}
\usepackage{cite}
\usepackage{rotating}
\usepackage{listings}
\usepackage{xcolor}
\usepackage{amsmath}
\usepackage{tikz}
\usepackage{enumitem}

\definecolor{codegreen}{rgb}{0,0.6,0}
\definecolor{codegray}{rgb}{0.5,0.5,0.5}
\definecolor{codepurple}{rgb}{0.58,0,0.82}
\definecolor{backcolour}{rgb}{0.95,0.95,0.92}

\lstdefinestyle{mystyle}{
    backgroundcolor=\color{backcolour},   
    commentstyle=\color{codegreen},
    keywordstyle=\color{magenta},
    numberstyle=\tiny\color{codegray},
    stringstyle=\color{codepurple},
    basicstyle=\ttfamily\footnotesize,
    breakatwhitespace=false,         
    breaklines=true,                 
    captionpos=b,                    
    keepspaces=true,                 
    numbers=left,                    
    numbersep=5pt,                  
    showspaces=false,                
    showstringspaces=false,
    showtabs=false,                  
    tabsize=2
}

\lstset{style=mystyle}
\usepackage{parskip}
\usepackage{setspace}
	\setstretch{1.25}



\title{\vspace{-10ex}Example Title}
\author{Peder Brandstorp Sanden}
\date{}
%\date{\vspace{-5ex}}  % Toggle commenting to test

\begin{document}
\maketitle

\section*{Section}
\subsection*{Some math}


  \begin{equation}
    T(n) = \begin{cases}
      0 & \text{if } n = 0\\
      1 & \text{if } n = 1\\
      T(n-1) + T(n-2) + 1 & \text{if } n \geq 2
      \label{defT(n)}
      \end{cases}
  \end{equation}

  \begin{equation}
    \begin{split}
      T(n) &= T(n-1) + T(n-2) + 1\\
      T(n) &= T(n-1) + T(n-2)\\
      T(n) &= 2 \cdot T(n-1)\\
      &\Downarrow \\
      T(n) &= 2 \cdot 2 \cdot T(n-2)\\
      T(n) &= 2^{k-1} \cdot T(n-k) \quad \{k = n-1\}\\
      T(n) &= 2^{k-1} \cdot T(1)\\
      T(n) &= O(2^n)
    \end{split}
  \end{equation}

\subsection*{A List}
The two properties a problem must have to be solved with dynamic programming are:
\begin{itemize}
  \item Optimal substructure
  \item Overlapping sub-problems
\end{itemize}

\section*{A Table}
Done after two iterations as there are no changes between iteration \textit{1.5} and \textit{2}, see table \ref{tab:2b}. The shortest path from \textit{s} to \textit{t} is 6 and goes: \textit{s-A-B-D-t}.

\begin{table}[h]
  \centering
  \begin{tabular}{c|c|c|c|c|c|c}
  \textbf{Iteration} & \textbf{s} & \textbf{A} & \textbf{B} & \textbf{C} & \textbf{D} & \textbf{t}\\
  \hline
  \textbf{0} & 0  & $\infty$  & $\infty$ & $\infty$ & $\infty$ & $\infty$\\
  \textbf{1} & 0 & 1 & 3 & 3 & 4 & 6\\
  \textbf{2} & 0 & 1 & 3 & 3 & 4 & 6\\
  \end{tabular}
  \caption{Iteratioins for finding the shortest path from s to t}
  \label{tab:2b}
\end{table}

\section*{Task 3: Maximum sum sub-array}
\begin{lstlisting}[language=Python]
def max_subarray_sum(A):
  max_ending_here = max_so_far = A[0]

  for num in A[1:]:
    max_ending_here = max(num, max_ending_here + num)
    max_so_far = max(max_so_far, max_ending_here)

  return max_so_far

# Example usage
example_array = [-2, 1, -3, 4, -1, 2, 1, -5, 4]
print(max_subarray_sum(example_array))
\end{lstlisting}
The time complexity of this is $O(n)$ as it only iterates through the array once.

\section*{Task 4: Grid traveling}
Traverse the grid from S to F by only moving right or down.

\subsection*{a)}
\begin{tabular}{|p{.3cm}|p{.3cm}|}
  \hline
  S &   \\
  \hline
    & F \\
  \hline
\end{tabular}\\\\
When the grid is 2x2 there are 2 ways to get from S to F.

\subsection*{b)}
\begin{tabular}{|p{.3cm}|p{.3cm}|p{.3cm}|}
  \hline
  S &   &   \\
  \hline
    &   &   \\
  \hline
    &   & F \\
  \hline
\end{tabular}\\\\
When the grid is 3x3 there are 6 ways to get from S to F.

\subsection*{c)}
\begin{tabular}{|p{.3cm}|p{.3cm}|p{.3cm}|p{.3cm}|}
  \hline
  S &   &   &   \\
  \hline
    &   &   &   \\
  \hline
    &   &   &   \\
  \hline
    &   &   & F \\
  \hline
\end{tabular}\\\\
When the grid is 4x4 there are 20 ways to get from S to F.
\subsection*{Observation}
The pattern can be recognized from table \ref{tab:pattern} where the number in the cells represents how many different ways one can traverse the grid from S to F with the specified rules.
\begin{table}[h]
  \centering
  \begin{tabular}{|p{.4cm}|p{.4cm}|p{.4cm}|p{.4cm}|}
    \hline
    1 & 1 & 1 & 1 \\
    \hline
    1 & 2 & 3 & 4 \\
    \hline
    1 & 3 & 6 & 10 \\
    \hline
    1 & 4 & 10 & 20\\
    \hline
  \end{tabular}
  \caption{Pattern}
  \label{tab:pattern}
\end{table}


\end{document}


